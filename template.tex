\documentclass[a4paper, 12pt, onepage]{article}
\usepackage[utf8]{inputenc}
\usepackage[russian]{babel}
\usepackage{fullpage}
\usepackage{indentfirst}
\usepackage{graphicx}
\usepackage{cmap}
\usepackage{amsmath}
\usepackage{amssymb}
%\usepackage{pdfpages}
\usepackage{tikz}

\usepackage[pdftex, unicode, pdfstartview=FitH, colorlinks, linkcolor=black, citecolor=blue, urlcolor=blue]{hyperref}
\usepackage{url}
\def\UrlFont{\rmfamily}

\usepackage{setspace}
\onehalfspacing

\usepackage{cyrtimes}
\renewcommand\ttdefault{cmtt}

\frenchspacing
\sloppy
\selectlanguage{russian}

\begin{document}

\author{Davidson, Scott}
\title{Debugging Using Resublimated Thiotimoline}
\maketitle

Silicon debugging is becoming more difficult with each generation of microprocessors.
The size, speed, and density of recent designs contribute to this complexity.
One major problem is failure latency. An error must propagate to an observable
point and then be observed; this may happen hundreds of cycles after the failure
occurs. The circuit's state can change significantly between a failure's occurrence
and observation, and evidence regarding the cause of failure can disappear.


\end{document}
